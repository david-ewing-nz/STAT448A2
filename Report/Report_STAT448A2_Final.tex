% !TEX TS-program = xelatex
\documentclass[11pt,a4paper]{article}
\usepackage[a4paper,margin=2.5cm]{geometry}
\usepackage{graphicx}
\usepackage{fancyhdr}
\usepackage{titlesec}
\usepackage{setspace}
\usepackage{fontspec}
\usepackage{parskip}
\usepackage{enumitem}
\usepackage{hyperref}

% Font setup
\setmainfont{Times New Roman}

% Header/Footer
\pagestyle{fancy}
\fancyhf{}
\rhead{STAT448 Assignment 2}
\lhead{David Ewing (82171165) \& Xia Yu (62380486)}
\rfoot{\thepage}

% Title spacing
\titlespacing\section{0pt}{12pt plus 2pt minus 2pt}{6pt plus 2pt minus 2pt}
\titlespacing\subsection{0pt}{10pt plus 2pt minus 2pt}{4pt plus 2pt minus 2pt}

% Title Page
\begin{document}

\begin{titlepage}
    \centreing
    \vspace*{3cm}
    
    {\Huge\bfseries STAT448 Assignment 2 -- Regression Modelling Across Domains\par}
    \vspace{1.5cm}
    
    {\Large\bfseries STAT448 – Assignment 2\par}
    \vspace{2cm}
    
    \begin{flushleft}
        \textbf{Submitted by:}\\
        David Ewing (82171165)\\
        Xia Yu (62380486)\\[1.5cm]

        \textbf{Due Date:}\\
        Friday 18th April 2025, 3:00 pm\\[1.5cm]

        \textbf{University of Canterbury}\\
        School of Mathematics and Statistics\\[1cm]
    \end{flushleft}

    \vfill
    \begin{flushright}
        \textit{“All work is our own. We have read and understood the university's policies on academic integrity.”}
    \end{flushright}
\end{titlepage}

% Set spacing for main document
\onehalfspacing

% -------------------------------
% Report Structure
% -------------------------------

\section*{1. Introduction}
\begin{itemize}
    \item Introduce the objective: predicting life expectancy using country-level indicators.
    \item Briefly describe the dataset and modelling goals.
    \item Outline the approach: modelling techniques, software (R), and reproducibility with seed 82171165 or 62380486.
\end{itemize}

\section*{2. Exploratory Data Analysis}
\begin{itemize}
    \item Summarise variable distributions and identify potential outliers.
    \item Examine correlations between predictors and the response.
    \item Include plots and discuss any data transformations or cleaning steps taken.
\end{itemize}

\section*{3. Modelling}
\begin{itemize}
    \item Fit at least three different models:
    \begin{itemize}
        \item Model 1: e.g., simple linear regression
        \item Model 2: e.g., GLM with transformations
        \item Model 3: e.g., ridge or lasso regression
    \end{itemize}
    \item For each model:
    \begin{itemize}
        \item Specify predictors and methodology
        \item Include and label relevant R code
        \item Present and interpret model results (coefficients, $R^2$, AIC, etc.)
        \item Check assumptions and include diagnostic plots
        \item Discuss strengths and limitations
    \end{itemize}
\end{itemize}

\section*{4. Model Selection}
\begin{itemize}
    \item Choose the most appropriate model for prediction.
    \item Justify the selection based on performance and assumptions.
    \item Discuss the trade-offs and practical implications of model choice.
\end{itemize}

\section*{5. Conclusion}
\begin{itemize}
    \item Summarise key findings from the modelling process.
    \item Reflect on limitations of the analysis.
    \item Suggest directions for further work or improvement.
\end{itemize}

\end{document}
